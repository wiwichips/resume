%!TEX TS-program = xelatex
%!TEX encoding = UTF-8 Unicode
% Awesome CV LaTeX Template
%
% This template has been downloaded from:
% https://github.com/posquit0/Awesome-CV
%
% Author:
% Claud D. Park <posquit0.bj@gmail.com>
% http://www.posquit0.com
%
% Template license:
% CC BY-SA 4.0 (https://creativecommons.org/licenses/by-sa/4.0/)
%


%%%%%%%%%%%%%%%%%%%%%%%%%%%%%%%%%%%%%%
%     Configuration
%%%%%%%%%%%%%%%%%%%%%%%%%%%%%%%%%%%%%%
%%% Themes: Awesome-CV
\documentclass[]{awesome-cv}
\usepackage{textcomp}
%%% Override a directory location for fonts(default: 'fonts/')
\fontdir[fonts/]

%%% Configure a directory location for sections
\newcommand*{\sectiondir}{resume/}

%%% Override color
% Awesome Colors: awesome-emerald, awesome-skyblue, awesome-red, awesome-pink, awesome-orange
%                 awesome-nephritis, awesome-concrete, awesome-darknight
%% Color for highlight
% Define your custom color if you don't like awesome colors
\colorlet{awesome}{awesome-red}
%\definecolor{awesome}{HTML}{CA63A8}
%% Colors for text
%\definecolor{darktext}{HTML}{414141}
%\definecolor{text}{HTML}{414141}
%\definecolor{graytext}{HTML}{414141}
%\definecolor{lighttext}{HTML}{414141}

%%% Override a separator for social informations in header(default: ' | ')
%\headersocialsep[\quad\textbar\quad]
    \begin{document}
    
%%%%%%%%%%%%%%%%%%%%%%%%%%%%%%%%%%%%%%
%     Profile
%%%%%%%%%%%%%%%%%%%%%%%%%%%%%%%%%%%%%%
\begin{center}
	\headerfirstnamestyle{} \headerlastnamestyle{Will Pringle} \\
	\vspace{2mm}
	{\faEnvelope\ willkantorpringle@gmail.com} | {\faGithub\ \href{https://github.com/wiwichips}{⠀github.com/wiwichips⠀}} | {\faLinkedin\ \href{https://www.linkedin.com/in/will-pringle/}{linkedin.com/in/will-pringle⠀}}
\end{center}
%%%%%%%%%%%%%%%%%%%%%%%%%%%%%%%%%%%%%%
%     Education
%%%%%%%%%%%%%%%%%%%%%%%%%%%%%%%%%%%%%%
\cvsection{Education}
\begin{cventries}
	\cventry
	{Bachelors in Software Engineering}
	{University of Guelph}
	{Guelph, Ontario}
	{Sep 2018 – April 2023}
	{GPA: 3.65}
\end{cventries}

\vspace{-2mm}
%%%%%%%%%%%%%%%%%%%%%%%%%%%%%%%%%%%%%%
%     Experience
%%%%%%%%%%%%%%%%%%%%%%%%%%%%%%%%%%%%%%
\cvsection{Experience}
\begin{cventries}
	\cventry
	{Incoming Software Developer}
	{Wave}
	{Toronto, Canada}
	{September 2021 – December 2021}
	{\begin{cvitems}
		\item {This fall I will be joining Wave as a GraphQL developer on the API team}
		\end{cvitems}}
  \cventry
	{Software Developer}
	{Kings Distributed Systems}
	{Kingston, Canada}
	{Jan 2020 – Present}
	{\begin{cvitems}
		\item {NodeJS, MySQL, Linux, HTML, CSS}
    \item {Core developer working on the Distributed Compute Protocol NodeJS microservices and MySQL database}
		\item {Implemented payment commissions, a feature that transfers a specified fraction of a payment to a separate account, allowing KingsDS to make revenue for every workload sent over their network}
		\item {Fixed critical race condition bugs in the network transport layer of the system improving security}
		\item {Re-branded a client facing frontend portal for a research study with over 100 participants using the Distributed Compute Protocol}
		\end{cvitems}}
	\cventry
	{Software Developer}
	{Ontario Ministry of Finance}
	{Toronto, Canada}
	{June 2020 – December 2020}
	{\begin{cvitems}
		\item {Python, R, Jupyter Notebook, SQL}
		\item {Automated data entry and modeling using Python reducing time for a critical task by 99\%}
		\item {Wrote an internal Python and R package to help economists analyze tax data used to write policy that affects millions of Ontarians}
		\item {Developed scripts to parse and analyze Excel data into a usable format using libraries such as Pandas and Numpy}
		\end{cvitems}}
\end{cventries}
\cvsection{Skills}
\begin{cventries}
	\cventry
	{}
	{\def\arraystretch{1.15}{\begin{tabular}{ l l }
		Programming Languages:  & {\skill{ JavaScript, C, Python, Kotlin, Java, Shell / Bash, SQL, R, Perl}} \\
		Technologies:  & {\skill{ Node.js, Linux / Unix, Android Studio, Linux Containers, Express.js, Tensorflow}} \\
		\end{tabular}}}
	{}
	{}
	{}
\end{cventries}

\vspace{-7mm}
\cvsection{Projects}
\begin{cventries}
	\cventry
  {\begin{cvitems}
  	\item{Utilized the Distributed Compute Protocol to distribute Clang x86 compilation of C code}
  \end{cvitems}}
	{JADCC (Distributed Clang Compiler)}
	{Webassembly, JavaScript, NodeJS}
	{https://github.com/wiwichips/jadcc}
	{}
	
	\vspace{-5mm}
	\cventry
  {\begin{cvitems}
    \item{Led a team to create an open source simple barcode scanning and inventory management system}
    \item{Implemented a frontend search system that queries a database for items previously scanned}
  \end{cvitems}}
	{ScanX}
	{Kotlin, Java, Android Studio, Python}
	{https://github.com/wiwichips/ScanX}
	{}

	\vspace{-5mm}
\end{cventries}
\cvsection{Awards and Extracurriculars}
\begin{cvhonors}
	\cvhonor
	{(First Place) Distribued Summer Hackathon}
	{Collaborated in a team of two under a 36 hour period to develop Gunkstribute, a project that utilizes DCP\textquotesingle{}s highly distributed supercomputer to classify images in a large photo gallery efficiently}
	{Distributed Compute Labs}
	{August 2020}
	\cvhonor
	{(Second Place) Amazon Day Datapath}
	{Competed to learn Amazon\textquotesingle{}s new query langugae Datapth to solve query related problems}
	{Amazon}
	{Jan 2019}
	\cvhonor
	{BCOMP Forum Student Organization - Founder}
	{• Grew an online community of over 1100 Computer Science students to run events and collaborate • Organized and ran a sponsored hackathon}
	{BCOMP}
	{September 2018 - Present}
\end{cvhonors}
\ 
\end{document}
